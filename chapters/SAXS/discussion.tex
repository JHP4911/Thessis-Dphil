%!TEX root = ../../main.tex
\section{Discussion}
\label{sec:Discussion}
\begin{itemize}
    \item Don't forget to mention the check about the first 3 frames in the results section.
    \item Cylindrical shapes is an explicit input option for RADDOSE-3D sample geometry. It converts a the description of the cylinder provided by the user, diameter and height to a polyhedral geometry description. In fact the current cuboid implementation does the same thing. Therefore currently the only shape that is modelled by RADDOSE-3D that isn't converted to a polyhedral representation for the simulation is the spherical representation.
    \item Doesn't yet account for samples flowed through the capillary.
    \item RADDOSE-3D modularity and what extensions are possible
\end{itemize}

\subsection{For Appendix?}
\label{sub:For Appendix?}
DATCMP is a software program, distributed as part of the ATSAS suite of software programs for SAXS data processing \cite{petoukhov2012new}, that assesses the similarity between SAXS frames.
The reduced $\chi^2$ statistic \cite{pearson1900x} was the standard method to do this and is given by:
\begin{equation}
    \chi^2 = \f{1}{n-1} \sum_{k=1}^{n}\left[ \f{I_i(q_k) - I_j(q_k)}{\sigma(q_k)}\right]^2
\end{equation}
where $n$ is the number of data points $k$, $I_i(q_k)$ and $I_j(q_k)$ are the intensities at the scattering vector direction $q_k$ for two different frames $i$ and $j$ (the scattering vector is defined in equation \ref{eq:momentum transfer}), and $\sigma(q_k)$ is the error at the $k^{th}$ data point \cite{franke2015synchrotron}.
$\chi^2 = 0$ implies that two scattering profiles are identical, $0.9 \le \chi^2 \le 1.1$ still suggests good agreement between two frames.
$\chi^2 >> 1$ suggests that two frames are very dissimilar.
The frame with the lowest dose is the first frame so it is assumed that the first frame is of the best quality.
Therefore all subsequent frames are compared to the first frame and any frames that show dissimilarity to the first are discarded for further data processing.
