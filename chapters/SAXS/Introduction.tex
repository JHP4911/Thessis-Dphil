%!TEX root = ../../main.tex
\chapter{Methods to assess radiation damage in SAXS}
\label{chap:Methods to assess radiation damage in SAXS}

 As outlined in the thesis introduction SAXS is a method to obtain a molecular envelope of a macromolecule in solution.
 However, during X-ray exposure the sample undergoes radiation induced changes which progresses throughout the experiment.
 Ultimately these changes lead to aggregation of the molecules in the sample which is observed as a progressive dissimilarity of subsequent 1D scattering curves.

 This chapter presents a quantitative study of possible mitigation of radiation damage by addition of various radioprotectant compounds in SAXS experiments. The compounds were chosen for several reasons:
 \begin{itemize}
     \item have shown good protection ability in MX \cite{allan2012} and \cite{southworth2007radioprotectant}.
     \item commonly used in already as radioprotectants in SAXS \cite{grishaev2012sample}.
     \item The majority of their atomic constituents have relatively small atomic numbers hence they will not significantly contribute to the absorbed dose in the experiment.
 \end{itemize}
