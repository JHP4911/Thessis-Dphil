%!TEX root = ../../main.tex
\chapter{Conclusions}
\label{chap:Conclusions}

\section{Radiation damage in MX}
\label{sec:Radiation damage in MX}
Radiation damage is the major cause of failed structure solution in MX.
In particular, global radiation damage, which is largely characterised by an overall decay of reflection intensities with increasing X-ray exposure, limits the amount of useful data that can be collected in a diffraction experiment.
The work presented in this thesis builds on existing research in MX for quantifying and correcting for global radiation damage.

\subsection{Extending the DWD metric}
\label{sub:Extending the DWD metric}
In a typical MX experiment, a crystal with a complex geometry is exposed to an X-ray beam with a non-uniform profile via an exposure strategy, which is sometimes non-standard, e.g. helical scanning.
Each of these factors can lead to inhomogeneous dose distributions within the crystal.
Zeldin \textit{et al.} developed the diffraction weighted dose (DWD), a dose metric designed to account for the dose distribution \cite{zeldin2013dwd}.
However, the DWD does not account for a loss in diffraction efficiency of a damaged crystal.
In this thesis, three dose decay models  (DDMs) were assessed for their ability to describe the relative intensity decay for data collected on several insulin crystals.
One of these models, \textit{RDE Leal}, which is a normalised form of the DDM proposed by Leal \textit{et al.} \cite{leal2012}, was used to represent the relative diffraction efficiency (RDE).
An $\eta$ term, which is a function of this RDE, was then added as a weighting term to the standard DWD to account for the effect of the loss in diffraction efficiency of a damaged crystal.
The various $\eta$ forms tested were:
\begin{itemize}
    \item $\eta$ = 1. (This is the same as the DWD originally developed in \cite{zeldin2013dwd}),
    \item $\eta$ = RDE (Decreasing function),
    \item $\eta$ = 1 - RDE (Increasing function).
\end{itemize}
Inclusion of the increasing and decreasing terms did not result in a decreased variability of the relative intensity as hypothesised by Zeldin \textit{et al.}, but this work does suggests that two metrics should be considered.
One metric that should represent the damaged state of the crystal, which should monotonically increase with increasing X-ray exposure.
This is typically the type of dose metric used in all radiation damage studies to date (maximum dose, which was reported by RADDOSE v1-3, average dose over the whole crystal and the DWD with $\eta$ = 1).
The other metric would represent the diffraction quality from the crystal using a decreasing $\eta$ form to weight the DWD.
This metric has more complicated behaviour and can decay back towards zero as the X-ray exposure increases.
In this case, images that result in similar DWD values should have similar diffraction quality, thus even images that are collected late in the experiment could have better diffraction quality than images collected in the middle of the experiment.
This suggests that this metric could be used, in addition to existing methods, to determine which images to merge to obtain good quality, reliable data.
To the author's knowledge, this type of metric has not been used before.

\subsection{Zero-dose extrapolation}
\label{sub:Zero-dose extrapolation}
The decay of reflection intensities as a result of global radiation damage can hinder structure solution by swamping the phasing signal in experimental phasing experiments.
Scaling algorithms attempt to correct for the overall intensity decay but they do not account for reflection specific intensity changes, which can be non-monotonic.
Previous studies have attempted to correct for the specific intensity decay (zero-dose extrapolation), showing that improvement in the phasing signal can be achieved \cite{diederichs2003,diederichs2006}.
However, zero-dose extrapolation is not commonly used \cite{borek2007many}, and is not that effective with low multiplicity data.
In this thesis, a zero-dose extrapolation model is presented that uses the Leal \textit{et al.} DDM for the traditional regression based extrapolation, which was able to capture the non-monotonic behaviour of reflection intensities.
The procedure also incorporated several quality checks to ensure the reliability of the regression fits.
Furthermore an additional probabilistic extrapolation routine was developed to perform zero-dose extrapolation for reflections with a small number of observations, which addresses the problem of performing extrapolation for low multiplicity data.
Further investigation of this method is required before it can be used reliably for any crystal.
In particular, it would benefit from the development of:
\begin{itemize}
    \item a method for learning the weight of the scaling factor used for the probabilistic extrapolation,
    \item an outlier rejection algorithm for poor observation extrapolations.
    \item a metric to assess the overall quality of the probabilistic extrapolation.
\end{itemize}

\subsection{Measuring X-ray beam profiles}
\label{sub:Measuring X-ray Beam profiles}
As mentioned above, the precise X-ray beam profile is a critical factor that determines the overall dose distribution in a crystal in an MX experiment.
The RADDOSE-3D dose calculation software has the capability to use experimentally measured beam profiles to accurately calculate the dose distribution.
Various methods for experimentally measuring the X-ray beam profiles exist, but each one requires different preprocessing before it faithfully represents the actual beam profile.
In this thesis, several methods for measuring beam profiles and processing the resulting data were investigated.
Aperture scans were performed at DLS beamline I02, where a piece of steel with a circular hole was translated across the X-ray beam, both vertically and horizontally, and measurements of the current from a diode was taken at regular intervals.
This only gives 1D data and therefore has to be converted to a 2D beam profile.
This was done by fitting a 2D Gaussian to the aperture data.
It was determined that deconvolving the resulting Gaussian profile with the aperture size did not change the overall beam profile significantly.
Another method used to measure the beam profile at the PETRA III beamline P14, was to use a scintillator combined with an Allied Vision GC1350C CCD camera, which resulted in a 2D image of the X-ray beam.
The advantage of this method is that it is not necessary to create a 2D profile.
The major disadvantage is that the image has background noise, which was shown to result in large errors in the dose calculation.
Various methods were explored to objectively remove the background, with some methods resulting in similar calculated $D_{1/2}$ values, a radiation damage sensitivity metric that represents the absorbed dose for which the relative intensity has decayed to half of its initial value.
However it is not objectively clear which method of removing the background is best, so the experimenter should use information that is known about the collimation of the beam to decide where the background is in an image.
Finally, the beam profile measurements carried out at ESRF beamline BM29, used several aperture scans, both horizonally and vertically, to obtain several 1D slices of the beam profile.
These data were then interpolated, as opposed to being modelled with a regular mathematical function.
The benefit of this approach is that the background does not have to be removed because the current actually decays to negligible levels.
It is also possible to represent and complex, irregular beam profiles.
For this reason it is recommended that beam profile measurements are performed by collecting several vertical and horizontal aperture scans.

\section{Data reduction}
\label{sec:Data Reduction}
Radiation damage correction has always focussed on attempting to correct the intensities that are observed in the diffraction experiment.
These intensities are ultimately generated by the contents of the crystal and so an alternative perspective to correcting for radiation damage is to describe (probabilistically) the changes that are occurring to the crystal.
Hidden Markov models are mathematical frameworks that track the hidden state of a system according to observations that are generated from it.
In this thesis we apply this framework to MX, where the state of the system is the crystal state represented by the set of structure factor amplitudes and the observations are the intensities.
The process and observation functions, along with their respective uncertainties (covariances), are defined as an exponential decay (temperature factor) and the product of the square of the amplitude with a scale factor respectively.
Applying the unscented Kalman filter (forward pass) with the unscented Rauch-Tung-Streibel smoother (backwards pass) is known as the forward-backward algorithm (FBA), which gives time resolved estimates of the amplitude values throughout the experiment.
Therefore, it is possible to use the amplitudes at each time point to run phasing and refinement to generate a set of electron density maps from a single dataset.
The set of amplitudes estimated at the first time point were then used for the successful structure determination of insulin and the C.Esp1396I protein-DNA complex by molecular replacement.
The resulting statistics used with this method are comparable to the statistics that result from structure solution using programs in the current data reduction pipeline (AIMLESS and CTRUNCATE).
In addition to producing time resolved amplitude estimates, the error estimates are also explicitly calculated in the forward-backward algorithm as a combination of the uncertainty in the measurement and the uncertainty in the (damage) process.
This means that the error estimates should be more representative of the expected error, which may lead to more reliable results in experimental phasing, but this is not verified and still needs to be investigated.

The HMM representation is not only useful for data reduction.
In fact the idea of the HMM in crystallography was spawned from the idea of modelling radiation damage in structure refinement (Garib Murshudov, personal communication).
The time resolved nature of the HMM means that refinement would directly lead to electron density movies.
The data reduction pipeline presented in this thesis is a successful proof of concept for the applicability of the HMM to crystallographic data.
Conceptually, the extension of the HMM to structure refinement is not too difficult.
Essentially the crystal state is no longer a set of structure factor amplitudes, but instead the crystal state is more accurately described by the set of structure factors.
The process function would need to be extended to incorporate a model for the change in atomic positions as well as the B factor.
In general the functions describing the B factor and the change in atomic positions are likely to be parameterised by the dose, either implicitly or explicitly.
However, for standard data collection where the crystal is rotated around a single axis and images are collected at regular intervals, this can be a function of the frame number, as was the case for the model described in this thesis.

\section{Quantifying radiation damage in SAXS experiments}
\label{sec:Quantifying radiation damage in SAXS experiments}
The work presented in this thesis presents a significant step forward in automating and exploring radiation damage in SAXS experiments.
Firstly, RADDOSE-3D has now been extended to allow dose calculations for SAXS experiments, just as easily as can be carried out for MX experiments.
Prior to this there was no open source, dose calculation standard for SAXS.
Furthermore, a Python library has been written to allow exploratory data analysis of SAXS datasets.
It provides a wrapper for executing DATCMP, which performs the similarity analysis described in section \ref{sub:Data analysis - experiment 2}, and data visualisation tools.
These tools were used to analyse datasets collected on glucose isomerase with different concentrations of radioprotectant compounds.
Glucose was found to be the most effective radioprotectant overall.
Additionally it was found that DTT exhibits odd behaviour, particularly at high concentrations.
One of the most intriguing visualisation tools in the Python library is the heatmap.
For DTT it showed regions of frame similarity which cannot be easily determined via other analytical methods.
This indicates that it may have implications for deciding which frames experimenters should merge to improve data quality, and possibly distinguish regions of different conformational states.
More work needs to be done to verify the importance of these regions.
