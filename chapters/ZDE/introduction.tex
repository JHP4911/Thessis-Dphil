%!TEX root = ../../main.tex
\chapter{Zero-Dose Extrapolation}
\label{chap:Zero-Dose Extrapolation}

\section{Introduction}
\label{sec:Introduction - Zero-Dose extrapolation}
Mathematical correction of reflection intensities to account for radiation damage has been a common procedure for decades \cite{hendrickson1973,abrahams1987anisotropy}.
Before the development of RADDOSE and RADDOSE-3D, the progression of radiation damage was tracked with time representing the independent variable.
Functions to describe the intensity loss took the form of the Blake and Phillips model \ref{eq:Blake and Phillips Radiation Damage Model} or isotropic polynomial expressions \cite{abrahams1987anisotropy}.

Radiation damage correction is still an integral part of current scaling methods \cite{otwinowski2003multiparametric,evans2005,kabsch2010integration}.
However, these correction models only compensate for the average decay of reflection intensities.
Account is not taken of specific changes in the correction algorithms.
Diederichs \textit{et al.}  introduced a linear model to perform zero-dose extrapolation, which was tested on data collected on bovine brain tubulin at 15$\,$K \cite{diederichs2003}, and later introduced exponential and quadratic models \cite{diederichs2006}, to account for these specific changes.
These authors showed that zero-dose extrapolation can improve phasing statistics and SAD electron density maps.
The problem is that reflections exhibit a diverse range of behaviours that are yet to be accounted for by a single Mathematical relationship \cite{blake1962,abrahams1973}.

Over the last decade there have been several developments in the methods and understanding of radiation damage in MX.
Some of the key developments in this field have been:
\begin{itemize}
    \item RADDOSE-3D \cite{zeldin2013}, which allows accurate 3D modelling of the diffraction experiment.
    \item the introduction of DWD \cite{zeldin2013dwd}, which is a useful metric for assessing the damage state of a crystal which resulted in the diffraction for a given image.
    \item the Leal \textit{et al.} dose decay model \cite{leal2012}. This relatively simple model was shown to accurately describe the intensity decay of reflections at RT.
\end{itemize}
The work described in this chapter is concerned with using the developments presented above to perform zero-dose extrapolation on data collected from one of the cubic crystals (crystal ID 0259) described in chapter \ref{chap:Dose Decay Modelling}.
