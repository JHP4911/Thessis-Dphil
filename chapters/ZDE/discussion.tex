%!TEX root = ../../main.tex
\section{Discussion}
\label{sec:Discussion - Zero-dose extrapolation}
As outlined in the introduction, global radiation damage in MX results in changes in the intensities of reflections throughout the experiment.
These intensity changes ultimately compromise the quality of the data and can affect the biological interpretations.
Crystallographers attempt to correct for these intensity changes during scaling by applying some form of overall damage correction.
These corrections do not account for specific intensity changes and can therefore introduce systematic errors \cite{diederichs2003}.

Zero-dose extrapolation is a method that attempts to correct for the specific intensity changes to improve data quality.
It has already been shown to improve phasing statistics and SAD electron density maps \cite{diederichs2003}.
However the linear model used in the 2003 study by Diederichs \textit{et al.} was only valid for a limited dose range, and fails to capture the variable behaviour of reflection intensity changes.
Quadratic and exponential models have also been explored with some success \cite{diederichs2006}, but ultimately the extrapolation is rarely used/effective (\cite{borek2007many}, Ed Lowe and Phil Evans, personal communication).

The work presented in this chapter takes advantage of a decade of additional radiation damage research to revisit zero-dose extrapolation.
The advances of particular interest to the research are the development of RADDOSE-3D, the DWD metric and the Leal \textit{et al.}  dose decay model.

Data from a uniformly irradiated insulin crystal (section \ref{sub:Data Collection and Dose Calculation}) were used to perform the extrapolation.
Dose calculations were carried out using RADDOSE-3D and the DWD metric was used as the independent variable in which to track the changes in reflection intensities.
In a similar manner to the approach used by Diederichs \textit{et al.}, some of the reflection intensities were extrapolated to zero-dose using regression analysis.
However, in the method presented here, multiple criteria had to be satisfied by a reflection before the extrapolation was carried out on it.
This was done to reduce the likelihood of overfitting the model to the data.
These included checking that the fits correlated well with the data, there were a minimum number of observations, the intensity observations were not too small and that the extrapolated intensity values did not give intensity values that were significantly different from the observed distribution.

To check the quality of the model fits and the zero-dose extrapolated intensity values, the $R_{fit}$ and $R_{zero}$ metrics were created.
Furthermore, the $R^{thresh}_{fit}$ and $R^{thresh}_{zero}$ metrics were used as thresholds to compare with the $R_{fit}$ and $R_{zero}$ and ensure that the overall regression extrapolations were reasonable.

Further to the regression based analysis, a probabilistic extrapolation using Bayesian inference was applied to the subset of data that was unsuitable for regression.
Using the data from the already extrapolated reflection observations, it was possible to estimate distribution parameters for the prior and likelihood distributions, which ultimately enabled estimates of the zero-dose intensity values for other observations.
This approach is new for zero-dose extrapolation and means that only a subset of the data are required to have a high enough multiplicity.
It is an important development because the low multiplicity of diffraction data is generally regarded as a major problem with the traditional regression based approach.

The extrapolation method presented here gave results consistent with that which was expected.
The Leal \textit{et al.} model was able to capture some of the non-linear and non-monotonic behaviour of the reflection intensities, and for the majority of reflections, the probabilistic method seemed to give reasonable zero-dose estimates.
In general the extrapolated intensities of reflections showed an increase from the observed values.

However, the method presented still requires some work before it can be used reliably for general crystallographic data.
The most obvious improvement is to find a method to prevent the probabilistic extrapolation method giving abnormally high intensity values.
As mentioned earlier, this issue could be corrected by using an outlier rejection method.
There is also currently no check for the quality of the extrapolation from the probabilistic approach, which would aid in highlighting potential problems.
A possible validation criteria may be that the probabilistic extrapolation gives values that are similar to the regression based extrapolation values for the reflections that have already been extrapolated.
This basically checks the consistency of the two methods.
The reason that the consistency check is a good way to validate that the probabilistic method works is that the regression based method is already validated with several checks and its own validation metrics.
The drawback with this is that the distribution parameters are calculated from the intensities extrapolated using the regression method.
This means that it is more likely that the two methods will seem consistent, but there is no guarantee that it will be reliable for the reflections that were not extrapolated using the regression approach.
Another approach to assessing the quality of the extrapolation is the using cross validation as is typically used in refinement in MX \cite{brunger199719}.
This involves keeping a relatively small subset of reflections aside and performing the extrapolation on the remaining data, then finally assessing the quality of the fits using all of the reflections at the end.
The potential problem with this approach is that there is generally not enough data to start with due to low multiplicity and the method requires a subset of reflections to have a sufficient multiplicity to perform the regression based extrapolation.
Therefore the ``free'' observation set may not be able to be picked completely randomly in all cases.

It is likely that the abnormally high extrapolated intensities are caused by applying the scale factor, $s(D)$, to reflections that do not obey the behaviour of the scale function.
$f(r)$ is a factor that was introduced to dampen the proportion of $s(D)$ that is applied to reflection observations.
But there are two problems.
Firstly, it is assumed that the similarity of the behaviour of a reflection to that of the scale factor is a function of the difference between the observed intensity value and the mean intensity value, $r$, in the appropriate resolution shell.
However, there is no apparent evidence to support this assumption, so it may not hold.
Secondly, the parameter values $A$ and $B$ in the equation \ref{eq:Fraction of scale factor to be applied - Extrapolation method} are not guaranteed to be optimal.
They were chosen by visual inspection of the behaviour of a limited set of observations and therefore were not rigorously determined.

The current method calculates the distribution parameter for the prior distribution using information obtained by spherically averaging the zero-dose extrapolated values. This should work fine when the data are isotropic, but could result in large errors when the data are anisotropic. A method that accounts for any potential anisotropy in the data is necessary to reduce these errors.

Furthermore, there are improvements that could be made to the regression based extrapolation.
It was observed that a ``bad'' correlation between the fitted model and the data was simply due to noise.
This implies that the correlation coefficient check may be too stringent, and in fact some reflections may be rejected despite the possibility that the model would have given a reasonable extrapolated zero-dose intensity.
Perhaps removing the correlation requirement in favour of a different criterion could improve the data quality by not rejecting ``well behaved'' reflections due to systematic noise.
The reason this was not done in the present analysis is because it was thought to be more important to have fewer reliable zero-dose estimates than a huge number where some estimates were questionable.

The corrected standard deviation equation (\ref{eq:Corrected sigma value - Extrapolation method}) is based on the form of the analogous equation in \cite{diederichs2003}.
Other forms of the correction function should be explored with a good physical/statistical justification.

A limitation of using the Leal \textit{et al.} model for the regression fits is that the model assumes positive intensity values for all (real) dose values.
As a physical model, this assumption is completely valid; however actual intensity data can be both positive and negative.
In the regression analysis, any reflection which had too many negative/small intensity values was rejected because this could lead to unreliable extrapolated intensity values.
Further work could include suggesting alternative physical dose decay models or amendments of the Leal \textit{et al.} model to satisfactorily account for negative intensity values.
However, it is currently reasonable to suggest that perhaps a single parametric dose decay model is not capable of describing the relationship between the change in intensity and the dose.
With this in mind, it could be the case that non-parametric regression techniques, such as Gaussian process regression, may prevail over parametric ones.
The problem is that non-parametric regression techniques often require more data for a reliable fit.
Furthermore, in my own experience, they are good for interpolation but not necessarily extrapolation.

The current implementation of the extrapolation algorithm was written in the Matlab programming language.
This was a suitable language to perform the data exploration necessary to do the analysis.
However the bottleneck with the program is reading the intensity data from the MTZDUMP output.
Matlab is over 1000 times slower than C for parsing integers \cite{bezanson2014julia}.
To make the program useable to the general crystallographic community, the program would need to be written in an open source and free programming language.
Python and C/C++ are currently the standard languages in the crystallographic community and would thus make good candidates.
