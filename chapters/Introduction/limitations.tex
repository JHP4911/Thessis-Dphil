%!TEX root = ../../main.tex
\section{Limitations of MX and SAXS and the alternatives}
\label{sec:Limitations and alternatives to MX and SAXS}
    Despite the dominance of MX as a method for structural analysis and the recent emergence of SAXS as a reliable method to provide complimentary structural and functional information, both methods have drawbacks.
    It has already been mentioned that producing protein crystals is a bottleneck of the pipeline in MX (section \ref{sub:Producing a protein crystal}).
    Even if a crystal can be grown it needs to be of a sufficient size and quality to give good quality diffraction.
    Often it is the case that there is a limit to the size and quality of the crystal that can be grown.
    Hence the hardware and software for data collection is constantly being developed to handle these situations.
    Theoretical work on the minimum crystal size for which sufficient data can be collected on a single crystal in a typical synchrotron radiation experiment \cite{holton2010} has inspired work on various synchrotron beamlines to achieve this limit.
    Another drawback of MX is the fact that it is essentially a method that only provides a temporal and spatial average of the molecules in the crystal, and hence no dynamical information is readily obtained from the experiment.
    However a relatively new technique making use of the Hadamaard transform is being developed that promises to give dynamic structural information about the molecule \cite{yorke2014time}.

    With regards to SAXS, the major disadvantage is that it does not provide atomic resolution information. Therefore relatively subtle structural dynamics of molecules will not be uncovered by using this technique.

    Several alternatives to MX and SAXS exist. These include X-ray free electron lasers (XFELs) and single particle cryo-electron microscopy (cryo-EM).
    In XFEL experiments, very intense beams of X-rays irradiate crystals over a very short time period (about 10$^{\text{12}}$ photons in a single pulse in about 10 femtoseconds (fs) \cite{chapman2011femtosecond}). XFELs can obtain high resolution diffraction information from nanocrystals.
    However the technique is still relatively immature.
    The data processing is not trivial (the crystals not being rotated during exposure and unknown distributions of X-ray beam wavelengths per pulse are just a couple of the factors complicating data processing), and access to XFELs is very limited.
    Cryo-EM, like SAXS, does not require the production of crystals but can additionally provide atomic resolution detail due to recent hardware and software advances \cite{bai2015cryo}.
    In cryo-EM a set of projection images of single molecules is taken with an electron microscope (after some sample preparation).
    The images are then computationally combined to obtain a density distribution of the molecule \cite{milne2013cryo}.
    Again, cryo-EM comes with its drawbacks. Obtaining high resolution structures from small (< 200-300$\,$kDa), unstable or flexible molecules is not trivial and it is not yet accessible enough \cite{bai2015cryo}.
