%!TEX root = ../../main.tex
\section{Radiation damage in SAXS}
\label{sec:Radiation damage in SAXS}
    The radiation damage literature for SAXS experiments is far less extensive than that for MX, and the phenomenon is less well understood.
    What is known is that the absorption events that occur in MX (primary damage) will occur in SAXS, since the physics of the interactions will not change.
    SAXS experiments are generally carried out at RT which means that in addition to electron radicals, hydroxyl (OH) and hydroperoxyl (HO$_2$) radicals created by the interaction of X-rays with water will also be mobile in the buffer solution \cite{jeffries2015limiting,garrison1987reaction}.
    These radicals attach to the protein backbone and/or sidechains which in turn induce protein aggregation through covalent or non-covalent bonding \cite{kuwamoto2004radiation}.
    The radiation damage is observed in the experimental data as a lack of overlap between 1D intensity curves for successive frames.

    There are several methods to mitigate the radiation damage caused in SAXS experiments. These include:
    \begin{itemize}
        \item The sample can be flowed through a capillary to limit the X-ray exposure for a given volume of sample.
        An alternative approach to this is to translate the capillary in the beam \cite{jeffries2015limiting}.
        \item In a similar manner to MX, the beam can be attenuated or defocussed \cite{jeffries2015limiting}.
        \item Radioprotectants such as DTT, TCEP, glycerol, ethylene glycol, ascorbate or sucrose can be added.
        These are thought to be able to capture hydroxyl radicals \cite{grishaev2012sample}.
        \item The SAXS experiment can be carried out at cryo-temperatures which reduces the damage per unit dose by up to five orders of magnitude compared to experiments at RT \cite{meisburger2013breaking}.
        However performing cryo-SAXS is not routine because the experimental set-up is far from trivial \cite{jeffries2015limiting}.
        It should be noted that dose estimates in SAXS experiments are calculated using equation \ref{eq:Dose Calculation 1D}, which is a one dimensional representation of the experiment.
    \end{itemize}

    Programs for primary data analysis do exist to assess the quality of the data \cite{petoukhov2012new}. In particular, DATCMP has recently been improved to provide correlation maps (CorMaps) to assess the similarity of the 1D scattering curves produced in a SAXS experiment \cite{franke2015correlation}. However, there is still room to improve the assessment of radiation damage progression online during the experiment.
