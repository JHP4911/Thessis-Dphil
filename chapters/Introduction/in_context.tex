%!TEX root = ../../main.tex
\section{This thesis in context}
\label{sec:This thesis in context}
	The work presented in this thesis extends previous work on radiation damage correction models.
	A new data reduction algorithm to process MX data has been developed that explicitly tracks the changes in the amplitude values throughout the diffraction experiment.
	The several benefits of this approach include:
    \begin{itemize}
        \item Explicit tracking of the amplitude uncertainties, even for negative and weak positive reflections.
		\item Estimates of amplitude values at every point in the diffraction experiment.
		\item The possibility to explicitly correct the amplitude values for radiation damage, as opposed to correcting the intensities.
    \end{itemize}
	Furthermore, RADDOSE-3D has been extended to calculate doses for SAXS experiments.
	This has been used to assess the efficacy of several radioprotectants for SAXS experiments.
	As a result of this work, a new metric has been developed for defining the threshold at which 1D SAXS curves are significantly damaged.
	\begin{itemize}
        \item \textit{Chapter 2} describes the experiment that was performed to collect data at 100\,K from several crystals of different proteins with a specially designed top-hat profile X-ray beam.
		These data are then used to test the validity of the decay models presented in section \ref{sub:Modelling intensity decay}.
		Finally the chosen model is used to improve the DWD metric proposed by Zeldin \textit{et al.}
		\item \textit{Chapter 3} describes the regression model used to make corrections to intensities of individual reflections using RADDOSE-3D and DWD.
		It also describes work which assesses the correlations of structure factor amplitudes derived from a small polypeptide with structural modifications.
		\item \textit{Chapter 4} reports the main results of this thesis. In this chapter, the new data reduction algorithm is presented.
		The results are also presented and compared to those using the current data reduction pipeline (AIMLESS \& CTRUNCATE).
		Considerations of the algorithm and extensions are also discussed.
		\item \textit{Chapter 5} describes work on assessing errors associated with calculating doses in RADDOSE-3D.
		This includes the handling of experimentally measured beam profiles with a significant background count and errors associated with uncertainties of crystal shape and other input parameters.
		The development of a graphical user interface for RADDOSE-3D is also discussed and how it can be used to compare the results of several irradiation strategies to improve experimental design.
		\item \textit{Chapter 6} describes the extension of RADDOSE-3D to calculate dose values for SAXS experiments.
		This extension is used to assess the dose suffered by samples in an experiment to determine the efficacy of radioprotectants in SAXS experiments.
		A new metric to define the threshold at which 1D SAXS curves are significantly damaged is also presented.
		\item \textit{Chapter 7} Summarises the work and conclusions drawn from the previous chapters and suggests future directions as a result of the work presented.
    \end{itemize}
