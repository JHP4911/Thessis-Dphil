%!TEX root = ../../main.tex
\chapter{Introduction}
\label{chap:Introduction - Intro chapter}
Determination of the three-dimensional (3D) structure of biological macromolecules is essential in developing our understanding of vital processes that occur within the cells of living organisms.
This is because the 3D structure of a macromolecule is one of the major factors that determine its function \cite{berg2002,hegyi1999}.
However, macromolecules are too small to be observed under a light microscope (the typical diameter for a soluble protein monomer is about 3-6$\,$nm \cite{Philips2015}, whereas light microscopes can only resolve objects that are at least a few hundred nanometres in size \cite{starr2010}) so alternative methods must be used to probe the molecular structure.
Several techniques to determine 3D molecular structures have been developed since the beginning of the 20th century.
The research presented in this thesis is concerned with methods development for the structural techniques of macromolecular X-ray crystallography (MX) and small angle X-ray scattering (SAXS) applied to protein molecules.
In particular, quantitative methods for assessing and correcting radiation damage for these techniques are explored.
Thus the following sections set the theoretical framework on which this work is based.
