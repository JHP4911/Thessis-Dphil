\fancyhf{} % clear all fields
\pagestyle{fancy}

%Define properties of the headers
\fancyhead[LE,RO]{{\color{black}%
\bfseries\thepage}}
\fancyhead[LO]{{\color{black}%
\bfseries\rightmark}}
\fancyhead[RE]{{\color{black}%
\bfseries\leftmark}}

%Define properties of the header line
\renewcommand{\headrulewidth}{2pt}
\renewcommand{\footrulewidth}{0pt}
\renewcommand{\headrule}{{\color{black}%
\hrule width\headwidth height\headrulewidth \vskip-\headrulewidth}}

%Set the indentation and distance between paragraphs
\setlength{\parindent}{0em}
\setlength{\parskip}{1em}

%Set the geometry of the page
\geometry{a4paper, inner=3.5cm, bottom=1.0cm, footskip=0.1cm}

%Set the headheight of the page. This was done to stop Latex moaning.
\setlength{\headheight}{15pt}

%This baselineskip gives sufficient line spacing for an examiner to easily
%markup the thesis with comments
%\baselineskip is a length command which specifies the minimum space between
%the botton of two successive lines in a paragraph.
\baselineskip=18pt plus1pt

%Set the number of sectioning levels that get number and appear in the contents
\setcounter{secnumdepth}{2}
\setcounter{tocdepth}{2}

%Set doublespacing
\doublespacing

\raggedbottom

%Bibliography style
\bibliographystyle{plain}

%Set Properties of chapter pages
\titleformat{\chapter}[display]
{\huge\filcenter}
{\titlerule[3pt]%
\vspace{3pt}%
\titlerule
\vspace{1pc}%
\LARGE\MakeUppercase{\chaptertitlename} \thechapter}
{1pc}
{\titlerule
\vspace{1pc}%
\Huge}
[\newpage] % creates the new page
